\documentclass[11pt,a4paper]{article}
\usepackage[T1]{fontenc} 
\usepackage[utf8]{inputenc}
\usepackage[english]{babel} 
\usepackage{verbatim}
\usepackage{graphicx}
\usepackage{acronym}
\usepackage{url}
\usepackage{cite}

\begin{document}

\section{Introduction}
\subsection{Background}
Large scale computer networks have been an uptrend in these years and will require a more scalable and reliable solution.\\
Traditionally, with periodically poll and centralized management system, some variables of the network, such as throughput, workload and bandwidth, are gathered through the whole network. Apart from several apparent drawbacks such as single point failure and performance bottleneck, centralized management approach usually cannot fulfill the need of mission-critical business processes, which will often need to deal with load changes and failure \cite{Stadler529980}.
Distributed algorithms are utilized to tackle these problems by propagating messages through the whole network in completely decentralized manner. Another advantage along with the solution is the local accessibility of global information \cite{jelasity_gossip-based_2005}.\\
Gossip-based algorithms, also known as epidemic-style techniques \cite{I.Gupta2006}, are used to deal with drawback of deterministic algorithms in exchange of certainties. On another hand, they are structureless algorithms, comparing to some other distributed protocols such as Echo algorithm proposed by Segell \cite{SegallG89}. Although, gossip-based algorithms could be combined with tree-based algorithms to reach an optimization for certain topologies \cite{KyasanurCG06}. Utilizing this algorithm, aggregation could be performed to collect data through the whole network and compute global information on every node.
Underlying topology has been proved by many works to have a strong impact on the performance of gossip-based algorithm \cite{5929538}\cite{jelasity_gossip-based_2005}, although little works exist regarding running gossip-based aggregation algorithm in real network topologies. This study would focus on investigating the divergence of performance in different real network topologies, namely Reuna, BREN, Geant, Iij \cite{knight_internet_2011}.



\end{document}