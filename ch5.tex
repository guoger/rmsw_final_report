\section{Conclusion \& Future work}
\label{sec:conclusion}

\subsection{Related work}
Motivated by peer-to-peer and ad hoc networks, a considerable number of studies have been done regarding gossip-based algorithms. Convergence and upper bound consensus time have been proven by J.Lavaei and R.Murray in \cite{5929538}. Analytical methods and simulations have been utilized to discuss the relations between performance of gossip protocols and topology of network, namely randomness, connectivity etc. As a subcategory of distributed algorithm different models, such as synchronous and asynchronous models, with or without churn, are discussed in \cite{Lynch:1996:DA:525656}. The optimization of parameters of an asynchronous randomized gossip algorithm for faster convergence is proven to be a semi-definite problem \cite{Boyd2004}.

This study is mainly based on the work of M.Jelasity, A.Montresor and O.Babaoglu \cite{jelasity_gossip-based_2005}, focusing on existing static networks of different topologies \cite{knight_internet_2011}

\subsection{Conclusion}

\subsection{Future work}
\subsubsection{More comprehensive emulation}
Since we assume an undirected unweighted static topology, some other crucial properties of a real network are not reflected in our study, such as link speed and congestions. Although, in real backbone networks, node leaving and joining are not likely to happen, but packet loss and delay could be modeled into churn mechanism and should be considered in future work.\\

\subsubsection{Relation between the performance and other properties of graph}
In \cite{jelasity_gossip-based_2005}, convergence factor $E(2^{-\phi})$ is used to determine convergence time, smaller convergence factor results in faster convergence. The Watts-Strogatz is used to model the topology of overlay network, indicating randomness as an independent variable \cite{Watts1998}. Thus, a function is derived, with randomness parameter $\beta$ as input and convergence factor $E(2^{-\phi})$ as output.\\
Although, to apply this model in a real network, reverse procedures need to be done to determine randomness parameter $\beta$ of a given graph $\mathcal{G}=\{\mathcal{E}, \mathcal{V}\}$. Mapping this back to the study of \cite{jelasity_gossip-based_2005} can give a better understanding how topology impact convergence time.\\
Other than this factor, graph clustering \cite{Schaeffer200727}, max-flow and min-cut \cite{PapadimitriouS82} could also be derived from these graphs and relationship between them and gossip algorithm performance can be further discussed.\\
