\section{Implementation}
\label{sec:implementation}
Before doing the actual experiment we did a preliminary analysis via simulation. The experiment contains of three components. First, the implementation of the gossip-based aggregation is clarified. Next, the network emulation context is presented. Finally it is shown under which conditions the experiment was run.

\subsection{Simulation of gossip-based aggregation}
\begin{figure}[!h]
\begin{lstlisting}[caption={Pseudo Code for gossip-based aggregation simulation}, label=code:simulation, mathescape=true, captionpos=b]
G = all_nodes
N(v) = neighbour_list_of_v
while(i <= Max_epoch)
	for n in G:
		m = random_choose(N(v))
		n.state = n.state + m.state
\end{lstlisting}
\end{figure}
\label{sec:simulation}
In a network simulation, the network is simplified so as to illuminate a certain set of behaviours (e.g. Network Simulator \cite{ns}). For the purpose of simulation, gossip-based aggregation in one epoch is simplified as a pair choosing process: in every epoch, every node is iterated exact once and one of its neighbour is randomly chosen to form a state exchange pair, and then aggregation operation is applied to them, which in this case is add-and-divide-by-two. Pseudo code can be found in listing \ref{code:simulation}. In this manner, convergence and different behaviours of four topologies can be predicted before running the actual emulation.

\begin{figure}[h!]
    \begin{center}
        \includegraphics[scale=0.25]{flow_diag.jpg}
    \end{center}
    \caption{Flow diagram of the threads}
    \label{fig:flow_diag}
\end{figure}

\subsection{Implementation of gossip-based aggregation daemon}
As the gossip agent in our emulated network we developed a daemon based on the pseudo code of the gossip-based aggregation algorithm (see section \ref{sec:theory}). The programming language Python \cite{python} was chosen for this task as its simplicity and clarity enable rapid development. Packets between nodes were exchanged in short lived TCP connections. The daemons on the individual nodes are started from outside the emulation system and write a log and results to the host system. After a number of epochs (successive fixed intervals) the daemons exit.

Two threads coexist (see figure \ref{fig:flow_diag}). The active thread connects to randomly chosen neighbours at an epoch duration. The passive thread accepts connection all the time. Each thread update the state of the node after a successful connection. To make the state develop linearly both threads have to compete for a lock before establishing a connection. %\footnote{Combining multithreading with networking led to some problems, especially when two nodes chose to actively create a connection with each other at the same time.}

Data created this way over one graph is analysed in R. We compare our four different topologies (mentioned above) and visualise the results as charts. The code of our program as well as the results below are available at github: \url{https://github.com/metaswirl/py_gossip}

\subsection{Emulation of gossip-based aggregation}
The aim of our experiment is to evaluate gossip based aggregation for real world computer networks. To this end the gossip-based aggregation has to be implemented over a computer network. Unfortunately the setup of a real network is too time and cost intensive. Hence the next best thing was chosen: A network emulation. 
An emulation consists of all the elements of the real network only that the resources are not "real" but "virtual". While for example an emulated switch may process network traffic in the very same way as a switch in your office, it could be implemented as a linux process.

\begin{figure}[h!]
    \begin{center}
        \includegraphics[scale=0.6]{graph_to_emulation}
    \end{center}
    \caption{From a graph to an emulated network}
    \label{fig:graph_to_emulation}
\end{figure}

In this research the Netkit emulation framework (itself a wrapper to user mode linux \cite{uml}) was chosen. Netkit \cite{netkit} creates virtual linux machines from a set of configuration files. These virtual machines are automatically connected through usermode processes called "uml\_switch". It is possible to log on to the machines via ssh and configure them just like any other linux machines. This allows great freedom to try and test software which otherwise would be hard to implement.

The initial set of configuration files were created with the tool Autonetkit \cite{autonetkit} using network graphs from the "Topology Zoo" \cite{knight_internet_2011}. The "Topology Zoo" is a collection of around two hundred fifty graphs of telecommunication networks. The graphs are open for use and give a good idea on the reality of communication networks. Simon Knight, one of the founders of the topology zoo has developed the tool Autonetkit. It takes a graph xml file (graphml) and constructs per host configuration files (from OSPF to DNS). For the purpose of this experiment Autonetkit was augmented to create for our gossip agent a file containing all neighbors of the individual nodes.

In summary real networks have been described in graphs at the topology zoo With the help of Autonetkit we create per host configuration files. These serve in turn as input for the user mode linux virtualization. See figure \ref{fig:graph_to_emulation} for a visualization of the four steps.

\subsection{Running the experiment}
The experiment was run on a host with 64 cores, each with 2299 MHz and 126GB of memory. Each virtual machine required 32Mbs of memory. The operating system was Ubuntu 11.10.

Every epoch lasted 15 seconds and the experiment was cut off after 100 epochs. These values have been found via trial and error.

The data is written by every node directly to disk into separate csv files. Those are later combined for statistical analysis in R.
