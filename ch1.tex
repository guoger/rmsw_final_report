\section{Introduction}
\label{sec:theory}
With the increasing interconnection of all sorts of devices their management too becomes increasingly difficult. Large scale computer networks require solutions more scalable and reliable than what conventional centralized management can provide.

Traditionally, variables of the network, such as throughput, workload and bandwidth, are gathered through the whole network by a central management system. Information is collected from all nodes of the network via periodic polling (e.g. SNMP) and stored in a database. Apart from several apparent drawbacks such as single point failure and performance bottleneck, a centralized management system cannot scale with an increasing number of nodes. In the long run this puts mission-critical business processes at risk \cite{Stadler529980}.

Distributed algorithms tackle these problems by propagating messages through the whole network in a more decentralized manner. Apart from dealing with the problems mentioned above they make global information locally accessible. Other components of the network can use this information to optimize their own behavior\cite{jelasity_gossip-based_2005}.

One variant of these algorithms is the Gossip-based aggregation. Gossip-based algorithms, also known as epidemic-style techniques \cite{I.Gupta2006}, are used to deal with drawback of deterministic algorithms. This algorithm is capable of aggregating global information in a completely decentralized manner. They differ from other distributed algorithms such as Echo algorithm proposed by Segell \cite{SegallG89} through their lack of structure. They do not require any structure of the network, other than an awareness of immediate neighbors. That said, gossip-based algorithms could be combined with tree-based algorithms to reach an optimization for some topologies \cite{KyasanurCG06}.

It has previously been proven that the underlying network topology has a strong impact on the performance of gossip-based algorithm \cite{5929538, jelasity_gossip-based_2003}. Yet little work exist investigating the behavior of gossip-based aggregation in real network topologies. This study focuses on the divergence of performance in different topologies of real computer networks. To this purpose a number of networks, sharing the same number of nodes, have been chosen from an online database \cite{knight_internet_2011}.

The remainder of the article is structured as follows. Section \ref{sec:theory} gives an introduction to the algorithm and our methodology. Next, section \ref{sec:implementation} presents the chosen implementation. Both the emulation of the networks and the implementation of the gossip algorithm are explained. Subsequently in section \ref{sec:results} the results of the experiment is presented. Conclusions and future work as well as related work is highlighted in section \ref{sec:conclusion}.
