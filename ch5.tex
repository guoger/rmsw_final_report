\section{Conclusion \& Future work}
\label{sec:conclusion}

\subsection{Conclusion}
% What we did
In this paper gossip-based aggregation was introduced. Its advantages as a distributed algorithm over centralized algorithms were explained and pseudo code was provided. Furthermore the experiment, which was supposed to show the dependency of convergence in gossip-based aggregation on the number of links in the network, was presented. The methods used to realize the experiment were highlighted and the choice for an emulated network was argued for. Finally the results of our research were discussed and depicted as charts.

% Our results
The data shows strong support for our hypothesis. The graphs with high number of links (Geant, Iij) converged quickly, while the graphs with low number of links converged slowly or not at all during the experiment. Gossip algorithm should be hence only used in densely connected networks or on tasks where slow information dissemination is acceptable. It should not be used on scarcely connected networks on time sensitive tasks. The expressivity of our experiment is however limited by the random nature of the algorithm. As this is only a initial study the experiment should be solidified via repetition.

% Implications
We believe that the gossip-based aggregation has very interesting properties for network management. The advantages would become more visible if instead of a toy value a more network relevant variable such as the general load distribution were used. Would such a value be available to each node the node could use the global information to optimize its own decisions. By simply changing the next-hop for outgoing packets it would already have a significant impact on the overall error rate of the network (drop rate of packets).

\subsection{Future work}
In our study, we only consider number of links yet there are many other attributes of a graph which could possibly impact aggregation performance such as graph clustering\cite{Schaeffer200727}, connectivity, randomness, max-flow min-cut etc. For one instance, although the difference between the number of links of {\it Reuna} and {\it Bren} is only two, the performance of both simulation and emulation varies significantly. If we look into the actual topologies of these two network, {\it Reuna} is a star-based topology without loops but {\it Bren} indicates a closer connectivity. Thus, connectivity could potentially impact the performance of aggregation, which need to be further verified.
On another perspective, since we assume an undirected unweighted static topology, some crucial properties of a real network such as link speed and congestion are not reflected by our study. Another possible extension would be to take node churn into account.

%\subsubsection{Relation between the performance and other properties of graph}
%In \cite{jelasity_gossip-based_2005}, convergence factor $E(2^{-\phi})$ is used to determine convergence time, smaller convergence factor results in faster convergence. The Watts-Strogatz is used to model the topology of overlay network, indicating randomness as an independent variable \cite{Watts1998}. Thus, a function is derived, with randomness parameter $\beta$ as input and convergence factor $E(2^{-\phi})$ as output.\\
%Although, to apply this model in a real network, reverse procedures need to be done to determine randomness parameter $\beta$ of a given graph $\mathcal{G}=\{\mathcal{E}, \mathcal{V}\}$. Mapping this back to the study of \cite{jelasity_gossip-based_2005} can give a better understanding how topology impact convergence time.\\
%Other than this factor, graph clustering \cite{Schaeffer200727}, max-flow and min-cut \cite{PapadimitriouS82} could also be derived from these graphs and relationship between them and gossip algorithm performance can be further discussed.\\
