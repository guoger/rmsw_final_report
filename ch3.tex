\documentclass[11pt,a4paper]{article}
\usepackage[T1]{fontenc} 
\usepackage[utf8]{inputenc}
\usepackage[english]{babel} 
\usepackage{verbatim}
\usepackage{graphicx}
\usepackage{acronym}
\usepackage{url}
\usepackage{cite}
\usepackage{amsmath}

\begin{document}
\newpage

\section{Chapter 3}
Implementation
* Topology Zoo
* Autonetkit
* Emulator
* Python Daemon

graphml -> configs -> uml machines
\section{Simulation/Emulation of gossip-based aggregation}
\subsection{Topology Zoo}
* "two hundred and fifty networks in the Zoo"
* communication networks of real telecommunication providers
* gallery and graphml files
* processed to remove multiple links between nodes (simplification)

\subsection{Autonetkit}
* By Simon Knight, also worked on topology zoo
* Translates graph file to linux configurations files (Qugga, Scripts, ...)
* multiple layers of abstraction
* we customized this so that every node stores a list of all it's neighbours and corresponding IP addresses

\section{Implementation of gossip-based aggregation protocol in virtual networks}
\subsection{Netkit & UML}
* Netkit a framework to do network experiments
* What is an emulation
* Uses User Mode Machines, the linux kernel is started as a userlevel process
* Combines kernel with a filesystem used mostly for pedagogic purposes
* network is created by using processes as switches between the linux kernel processses

pseudo code -> python scripts -> csv -> R scripts -> graphs
\subsection{Gossip Daemon}
* written in python
* based on a modular structure of socket, epoch and state modules
* difficulties in combining multithreading with network

\subsection{R scripts}
...

\end{document}
